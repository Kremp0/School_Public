\documentclass[11pt,a4paper]{article}

\usepackage[left=2cm,text={17cm, 24cm},top=3cm]{geometry}

\usepackage[czech]{babel}
\usepackage[utf8]{inputenc}
\usepackage{times}
\usepackage[IL2]{fontenc}

\usepackage{url}

\bibliographystyle{czplain}

\begin{document}

	\begin{titlepage}
		\begin{center}
			\Huge
			\textsc{\Huge Vysoké učení technické v~Brně \\ {\huge Fakulta informačních technologií \\}}
			\vspace{\stretch{0.382}}
			{\LARGE Typografie a publikování --\ 4. projekt} \\ {\Huge Bibliografické citácie}
			\vspace{\stretch{0.618}}
		\end{center}
		{\Large \today \hfill
			Róbert Kolcún, xkolcu00}
	\end{titlepage}
	
	
	%----------------------NEWPAGE-------------------
	\section{\LaTeX}
	
		\LaTeX je vysoce kvalitní typografický systém určený pro profesionální a poloprofesionální sázení dokumentů \cite{odkaz:Martinek}.
		
		LaTeX používa jako formátovací jazyk sázecí systém TeX - značkovací jazyk. Je založen na myšlence, že autor dokumentu by se měl starat pouze o~text článku, zatímco o~formátování sa postarají vývojáŕi dokumentu \cite{kniha:Rybicka} Je vhodný pro zápis matematických vzorců, tvorbu knih, bakalářských \cite{bakalarka:Korekce} a diplomových prací \cite{diplomovka:Znakovesady}, umožňuje vytváŕení vlastních maker, které jsou vhodné např. na změnu vzhledu celého dokumentu.
	
	\section{Font}
	
		Pojem font se využívá především v~typografii, kde je definován jako kompletní sada znaků abecedy jedné velikosti a jednotného stylu \cite{prop:Computing}.
		Počítačová písma jsou digitální elektronická data, která obsahují jednotlivé znaky \cite{kniha:Friedl}. Dělí se do tří základních druhů:
	
		\begin{enumerate}
			\item[$\bullet$] \textbf{Rastrová} písma se skládají z~několika pixelů, které představují obrázky jednotlivých znaků v~každém písmu jakékoliv velikosti \cite{odkaz:Microsoft}.
			\item[$\bullet$] \textbf{Konturová} písma (nazývaná též vektorová) používají Bézierovy křivky, kreslící instrukce a matematické vzorce k~popisu jednotlivých znaků, což umožňuje škálovatelnost těchto znaků do všech velikostí.
			\item[$\bullet$] \textbf{Čárová} písma používají sérii specifických čar a doprovodných informací k~utvoření profilu nebo velikosti a tvaru čáry ve specifickém písmu, což dohromady utváří celý znak.
		\end{enumerate}
	
	\section{Programovanie}
	
		Programování je proces zahrnující činnosti od návrhu algoritmu, psaní, testování a ladění zdrojového kódu počítačového programu (software), včetně následné údržby \cite{kniha:MacLennan}.
		Cílem programování je vytvořit program, který vykazuje určité žádoucí chování. Proces psaní zdrojových kódů často vyžaduje odborné znalosti v~mnoha různých oborech, včetně znalosti oblasti použití \cite{odkaz:Foundation}.
		
		Pod pojmem programovací jazyk rozumíme prostředek pro zápis algoritmů, jež mohou být provedeny na počítači. Je komunikačním nástrojem mezi programátorem, který v~programovacím jazyce formuluje postup řešení daného problému, a počítačem, každý programovací jazyk má pravidla podle kterých funguje a programátor se na ně může spolehnout \cite{prop:Normy}.
		
	%----------------------NEWPAGE-------------------
	\newpage
	\bibliography{literatura}

		
\end{document}